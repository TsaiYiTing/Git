\documentclass{beamer}

% for themes, etc.
\mode<presentation>
%\usetheme{Warsaw}
%\usecolortheme{dolphin}
%{\usetheme{Singapore}}
%{ \usetheme{lined} }
{\usetheme{Dresden}}
\usepackage{fontspec}   %加這個就可以設定字體
\usepackage{xeCJK}       %讓中英文字體分開設置
\setCJKmainfont{標楷體} %設定中文為系統上的字型,而英文不去更動,使用原TeX字型
\XeTeXlinebreaklocale "zh"             %這兩行一定要加,中文才能自動換行
\XeTeXlinebreakskip = 0pt plus 1pt     %這兩行一定要加,中文才能自動換行
\usepackage{enumerate}%列點序號
\usepackage{amsmath,amssymb,amsfonts,booktabs}
\usepackage{epic}
\usepackage{mathpazo}  % fonts are up to you
%\usepackage{astats,epsfig}
\usepackage{graphicx,epsfig,subfigure}
\usepackage{pdftexcmds}
\usepackage{xcolor}
\usepackage{float}
\usepackage{url} %超連結
\def\UrlFont{\tt}

\newtheorem*{remark}{Remark}
\setlength{\parindent}{0mm}%第一段不縮排
\setlength{\parskip}{10pt}  % 每段間隔

\newtheorem{thm}{\bf{Theorem.}}
\renewcommand{\proofname}{\ctxfk \textbf{Proof.}}
%\usepackage{array,booktabs}
% these will be used later in the title page
\title{Convolutional Neural Networks}
\author{Yi-Ting Tsai}
\institute{National Sun Yat-sen University}
\date{2018/10/24}
\begin{document}
		\begin{frame}
		\titlepage
        \end{frame}

        \begin{frame}
		\frametitle{Outline} % make a frame titled "Outline"
		\tableofcontents % show TOC and highlight current section
        \end{frame}
\section{Review-example}
    \begin{frame}
    \frametitle{Why CNN for Image: }
        \begin{itemize}
            \item[.] Some patterns are much smaller than the whole image
            \item[.] The same patterns appear in different regions.
            \item[.] Subsampling the pixels will not change the object
        \end{itemize}
    \end{frame}

    \begin{frame}
    \frametitle{Example: }
        \begin{figure}[H]
            \begin{center}
                \includegraphics[width=7cm]{ppt1}
            \end{center}
        \end{figure}
    \end{frame}

    \begin{frame}
    \frametitle{Example: }
        \begin{figure}[H]
            \begin{center}
                \includegraphics[width=7cm]{ppt2}
            \end{center}
        \end{figure}
    \end{frame}
    \begin{frame}
    \frametitle{Example: }
        \begin{figure}[H]
            \begin{center}
                \includegraphics[width=7cm]{ppt3}
            \end{center}
        \end{figure}
    \end{frame}
    \begin{frame}
    \frametitle{Example: }
        \begin{figure}[H]
            \begin{center}
                \includegraphics[width=7cm]{ppt4}
            \end{center}
        \end{figure}
    \end{frame}
    \begin{frame}
    \frametitle{Example: }
        \begin{figure}[H]
            \begin{center}
                \includegraphics[width=7cm]{ppt5}
            \end{center}
        \end{figure}
    \end{frame}
    \begin{frame}
    \frametitle{Example: }
        \begin{figure}[H]
            \begin{center}
                \includegraphics[width=7cm]{ppt6}
            \end{center}
        \end{figure}
    \end{frame}
    \begin{frame}
    \frametitle{Example: }
        \begin{figure}[H]
            \begin{center}
                \includegraphics[width=7cm]{ppt7}
            \end{center}
        \end{figure}
    \end{frame}
    \begin{frame}
    \frametitle{Example: }
        \begin{figure}[H]
            \begin{center}
                \includegraphics[width=7cm]{ppt8}
            \end{center}
        \end{figure}
    \end{frame}
    \begin{frame}
    \frametitle{Example: }
        \begin{figure}[H]
            \begin{center}
                \includegraphics[width=7cm]{ppt9}
            \end{center}
        \end{figure}
    \end{frame} 
\section{Pooling Layer}
    \begin{frame}
    \frametitle{Pooling Layer: }
        Their  goal  is  to  \emph{subsample}  (i.e.,  shrink)  the  input  image  in  order  to
        reduce  the  computational  load,  the  memory  usage,  and  the  number  of  parameters
    \begin{figure}[H]
            \begin{center}
                \includegraphics[width=7cm]{FIGURE13-8}
            \end{center}
        \caption{\emph{Max pooling layer}(2 × 2 pooling kernel, stride 2, no padding)}
        \end{figure}
    \end{frame}
\section{CNN Architectures}
    \begin{frame}
    \frametitle{Typical CNN architectures: }
        \begin{figure}[H]
            \begin{center}
                \includegraphics[width=7cm]{FIGURE13-9}
            \end{center}
        \caption{Typical CNN architecture}
        \end{figure}
    \end{frame}

    \begin{frame}
    \frametitle{CNN Architectures: }
        \par A good measure of this progress is the error rate in  competitions  such  as  the ILSVRC  ImageNet  challenge.
             In  this  competition  the top-5 error rate for image
             classification fell from over $0.26$ to barely over $0.03$ in just five years.
        \par We will first look at the classical LeNet-5 architecture (1998), then three of the winners
             of  the  ILSVRC  challenge:  AlexNet  (2012),  GoogLeNet  (2014),  ResNet(2015) , and SENet(2017) .
    \end{frame}

    \begin{frame}
    \frametitle{CNN Architectures: }
        \begin{figure}[H]
            \begin{center}
                \includegraphics[width=7cm]{FIGURE13-1}
            \end{center}
        \caption{ILSVRC  ImageNet  challenge}
        \end{figure}
    \end{frame}
\subsection{LeNet-5 architecture}
    \begin{frame}
    \frametitle{LeNet-5 architecture: }
            \begin{figure}[H]
            \begin{center}
                \includegraphics[width=7cm]{table13-1}
            \end{center}
        \caption{LeNet-5 architecture}
        \end{figure}
    \end{frame}

\subsection{AlexNet architectures}
    \begin{frame}
    \frametitle{AlexNet architectures: }
        \begin{figure}[H]
            \begin{center}
                \includegraphics[width=7cm]{table13-2}
            \end{center}
        \caption{AlexNet architectures}
        \end{figure}
    \end{frame}

    \begin{frame}
    \frametitle{AlexNet architectures: }
        \begin{figure}[H]
            \begin{center}
                \includegraphics[width=7cm]{FIGURE13-2.png}
            \end{center}
        \caption{AlexNet architectures}
        \end{figure}
    \end{frame}
\subsection{GoogLeNet architectures}
    \begin{frame}
    \frametitle{GoogLeNet architectures: }
         \begin{figure}[H]
            \begin{center}
                \includegraphics[width=7cm]{table13-3}
            \end{center}
        \caption{GoogLeNet architectures}
        \end{figure}
    \end{frame}

    \begin{frame}
    \frametitle{GoogLeNet architectures: }
         \begin{figure}[H]
            \begin{center}
                \includegraphics[width=3cm]{FIGURE13-3.png}
            \end{center}
        \caption{GoogLeNet architectures}
        \end{figure}
    \end{frame}

\subsection{ResNet architectures}
    \begin{frame}
    \frametitle{ResNet architectures: }
         \begin{figure}[H]
            \begin{center}
                \includegraphics[width=7cm]{FIGURE13-4.png}
            \end{center}
        \caption{ResNet architectures}
        \end{figure}
    \end{frame}

    \begin{frame}
    \frametitle{ResNet architectures: }
         \begin{figure}[H]
            \begin{center}
                \includegraphics[width=7cm]{FIGURE13-5.png}
            \end{center}
        \caption{ResNet architectures}
        \end{figure}
    \end{frame}

    \begin{frame}
    \frametitle{ResNet architectures: }
         \begin{figure}[H]
            \begin{center}
                \includegraphics[width=7.5cm]{FIGURE13-6.png}
            \end{center}
        \caption{ResNet architectures}
        \end{figure}
    \end{frame}

    \begin{frame}
    \frametitle{ResNet architectures: }
         \begin{figure}[H]
            \begin{center}
                \includegraphics[width=7cm]{FIGURE13-7.png}
            \end{center}
        \caption{ResNet architectures}
        \end{figure}
    \end{frame}

\subsection{SENet architectures}
    \begin{frame}
    \frametitle{SENet architectures: }
    \end{frame}

   \begin{frame}
   \center{\huge{Thanks for listening}}
   \end{frame}
\end{document}        